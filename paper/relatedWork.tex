%!TEX root = main.tex
%%%%%%%%%%%%%%%%%%%%%%%%%%%%%%%%%%%%%%%%%%%%%%%%%%%%%%%%%%%%%%%%%%%%%%%%%%%%%%%%

\section{Introduction}

How helpful is the classification of games into genres? What purpose does it accomplish?

How do commercial platforms classify games? Their goal is selling many games. Video game platforms like steam tag games so that players can search for a game they want to buy. 

While there does not exist a refined model for the QoE of games, mayor impact factors are well known. These impact factors include:
\begin{itemize}
\item frame rate
\item graphical fidelity
\item 
\end{itemize}

There exist various user studies that inspect a specific aspect.

When mapping from QoS parameters on a technical level such as network or hardware to the QoE of Gaming, there are two steps. First, you have to consider the impact of the technical QoS on the QoS on the application layer. Second we consider the impact of the game QoS on the QoE of a gaming session and many game play factors that are discussed in detail in [related work].

QoS parameters on a techincal level may include the CPU power, the network delay, the tick rate, etc ...
QoS parameters on application layer may be very game specific and include the total delay (command input to display on the monitor), the game resolution, the game frame rate.

In \cite{jarschel2011evaluation}, the authors look at QoS parameters of the network such as the network delay and the packet loss and analyze their impact on the QoE. They categorize game by one dimension (pacing): fast, medium and slow. However, they do not investigate QoS parameters on the application level.

In \cite{claypool2006effects}, the authors investigate the impact of QoS on application level such as frame rate and resolution on the QoE. Their study is based on the game Quake 3.

In \cite{pathania2014integrated}, the authors investigate the impact of hardware based QoS parameters such as cpu, gpu, power consumption on the frame rate

In \cite{claypool2006latency}, the authors 

